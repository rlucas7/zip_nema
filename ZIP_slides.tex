\documentclass{beamer} % "Beamer" is a word used in Germany to mean video projector. 

\graphicspath{{figs/}}

\usetheme{Berkeley} % Search online for beamer themes to find your favorite or use the Berkeley theme as in this file.

\usepackage{color} % It may be necessary to set PCTeX or whatever program you are using to output a .pdf instead of a .dvi file in order to see color on your screen.
\usepackage{graphicx} % This package is needed if you wish to include external image files.

\theoremstyle{definition} % See Lesson Three of the LaTeX Manual for more on this kind of "proclamation."

\title{Why Scientists Should care about zero-inflation and what to do about it}
\author{Lucas Roberts} 
\institute{The Hartford Insurance}
%\date{January 6, 2012} 
% Remove the % from the previous line and change the date if you want a particular date to be displayed; otherwise, today's date is displayed by default.

\AtBeginSection[]  % The commands within the following {} will be executed at the start of each section.
{
\begin{frame} % Within each "frame" there will be one or more "slides."  
\frametitle{Presentation Outline} % This is the title of the outline.
\tableofcontents%[currentsection]  % This will display the table of contents and highlight the current section.
\end{frame}
} % Do not include the preceding set of commands if you prefer not to have a recurring outline displayed during your presentation.

\begin{document}

\begin{frame} 
\titlepage
\end{frame}

\section{Problem Background} % Since this is the start of a new section, our recurring outline will appear here.

\begin{frame} 
\frametitle{Excess zeroes}

\begin{columns} % This creates a frame with multiple columns.
\begin{column}{0.5\textwidth} % The first column will be 50% as wide as the width of text on the page.
Example\\
\pause 
$n=100$ observations\\
\pause
average of $\bar{y} = 2$ \\
\pause 
expect $100\times e^{-2}$ \\
\pause 
$\approx 13.5$ zeroes 
\pause
\end{column}
\begin{column}{0.5\textwidth} % The first column will be 50% as wide as the width of text on the page.
\begin{figure}[htb]
\includegraphics[width=0.85\textwidth]{rn.jpg}
\end{figure}
\end{column}
\end{columns}
\end{frame}

\begin{frame} 
\frametitle{But what if..? }
\begin{columns} % This creates a frame with multiple columns.
\begin{column}{0.5\textwidth} % The first column will be 50% as wide as the width of text on the page.
Example\\
$n=100$ observations\\
average of $\bar{y} = 2$ \\
expect $100\times e^{-2}$ \\
$\approx 13.5$ zeroes 
\end{column}
\begin{column}{0.5\textwidth} 
You observe: \\
\pause
$n=25$ zeroes \\
\pause
or $n=50$ zeroes \\
\pause
or $n=75$ zeroes \\
\end{column}
\end{columns}
\end{frame}

\begin{frame} 
\frametitle{Now what? }
1. We need a method to determine how many zeroes are too many and \\
\vspace{1in}
\pause
2. Need a model that accounts for the excess-or scarcity-of zeroes
\end{frame}

\begin{frame}
\frametitle{Take Home Points}
1. Failing to account for excess zeroes  leads to underestimates (negative bias) of rates. \\
\pause 
2. You will never overcome this bias by solely taking more measurements. \\
\pause 
3. The fix is (relatively) easy, adjust the model! \\
\end{frame}

\section{Counting the Species} 

\begin{frame} 
\frametitle{Some history}

\begin{columns} % This creates a frame with multiple columns.
\begin{column}{0.5\textwidth} % The first column will be 50% as wide as the width of text on the page.
Looking at nematode count data for about 3 years\\
\pause
My collaborator likes to take measurements \\
\pause
We keep seeing zero counts and lots of interesting ways to adjust for excess zeroes\\
\end{column}
\begin{column}{0.5\textwidth} % The first column will be 50% as wide as the width of text on the page.
\begin{figure}[htb]
\includegraphics[height=1\textwidth]{soil2.jpg}
\end{figure}
\end{column}
\end{columns}
\end{frame}

\begin{frame} 
\frametitle{How did I get here II}
``Everybody knows you just add 1 to everything''\\ 
\pause
\vspace{0.5in}
``$log(1+\text{count})$ and then take an average''\\
\pause
\vspace{0.5in}
``$\sqrt{2\times \text{count}}$ and then take average''\\
\pause 
\vspace{0.5in}
condition on count $>0$ and then take averages\\
\end{frame}

\begin{frame} 
\frametitle{How did I get here II}
\begin{columns} % This creates a frame with multiple columns.
\begin{column}{0.4\textwidth} % The first column will be 50% as wide as the width of text on the page.
Not one scientist mentioned\\
a zero-inflated model. \\
\vspace{0.5in}
\pause
Sometimes the \\
diffusion of \\
knowledge takes \\
time. 
\end{column}
\begin{column}{0.5\textwidth} % The first column will be 50% as wide as the width of text on the page.
\begin{figure}[htb]
\hspace{-0.4in}\includegraphics[trim={1cm 0 4cm 1cm}, height=1\textwidth]{gill_soil.jpg}
\end{figure}
\end{column}
\end{columns}
\end{frame}

%Some nematode species are vectors of diseases for plants spreading blight and other pernicious disease. Currently they are impacting rice yields in Cambodia by 30\%. 

\begin{frame}
\frametitle{Measurements}
\begin{columns} % This creates a frame with multiple columns.
\begin{column}{0.5\textwidth} % The first column will be 50% as wide as the width of text on the page.
Desire to accurately assess the rate of each species in a soil sample. \\
\vspace{1in}
\pause
Rates are used to determine when a region is experiencing an outbreak. \\
\pause
\end{column}
\begin{column}{0.5\textwidth} % The first column will be 50% as wide as the width of text on the page.
\begin{figure}[htb]
\hspace{-0.4in}\includegraphics[trim={1cm 0 4cm 1cm}, height=1\textwidth]{elutriator.png}
\end{figure}
\end{column}
\end{columns}
\end{frame}

\begin{frame}
\frametitle{The data from a soil lab}
\begin{columns} % This creates a frame with multiple columns.
\begin{column}{0.5\textwidth}
For each sample you have a date of extraction
\vspace{0.25in}
\pause 
and any specifics of that sample\\
\pause
\vspace{0.25in}
Then the counts for each specie
\vspace{0.25in}  
\pause
\end{column}
\begin{column}{0.5\textwidth} % The first column will be 50% as wide as the width of text on the page.
\begin{figure}[htb]
\hspace{-0.4in}\includegraphics[trim={1cm 0 4cm 1cm}, height=1\textwidth]{elutriator.png}
\end{figure}
\end{column}
\end{columns}
% The command \uncover<m->{STUFF} means that STUFF will appear starting in the mth slide of the frame.
% The command \uncover<m-n>{STUFF} means that STUFF will appear from the mth slide to the nth slide of the frame.
\end{frame}

\begin{frame}
\frametitle{Data}
\begin{figure}[htb]
\includegraphics[scale=0.5]{data_2016_nematode.png}
\end{figure}
\end{frame}

% Since this is the start of a new section, our recurring outline will appear here.

\section{Statistical aspects} 
\begin{frame}
\frametitle{Distributions for count data I}
\begin{columns} % This creates a frame with multiple columns.
\begin{column}{0.5\textwidth}
Often a Poisson \\
distribution is \\
used for count data

$ \Pr(Y=y)=
\exp(-\lambda)\lambda^Y/Y!, \text{if }y\geq0 
$

\end{column}
\begin{column}{0.5\textwidth} % The first column will be 50% as wide as the width of text on the page.
\begin{figure}[htb]
\hspace{-.45in}\includegraphics[scale=0.35]{pois_pmf.pdf}
\end{figure}
\end{column}
\end{columns}
\end{frame}

\begin{frame}
\frametitle{Distributions for count data II}
Typically a zero-inflated Poisson distribution is used for counts with many zeroes

$$ \Pr(Y=y)=
\begin{cases}
(1-\phi)\exp(-\lambda)\lambda^Y/Y!, &\text{if }y\neq0 \\
\phi+(1-\phi)\exp(-\lambda), &\text{if }y=0
\end{cases}
$$
Here $1 > \phi > 0$ and $\lambda >0$.
\end{frame}

\begin{frame}
\frametitle{Distributions for count data II(cont)}
\begin{figure}[htb]
\hspace{-.45in}\includegraphics[scale=0.35]{zi_pois_pmf.pdf}
\end{figure}
\end{frame}


\begin{frame}
\frametitle{Distributions for count data III}
We include a Bernoulli to distinguish amongst zeroes. \\ 
\vspace{.5in}
$ \Pr(Y=y, Z=z)=
\phi^Z\left[(1-\phi)\exp(-\lambda)\lambda^Y/Y! \right]^{1-Z} 
$ \vspace{.5in}
Here $1 > \phi > 0$, $\lambda >0$, $y \geq 0$, and $Z \in \{0,1\}$. 
If $y>0$ then $Z\equiv0$. 
\end{frame}

\begin{frame}
\frametitle{Bias}
What if the data are really zero-inflated? Then, \\
\pause
$\mathbb{E}(\bar{y}) = (1-\phi)\lambda$ and the bias is: \\
\pause
$\text{bias} = -\phi\lambda$.\\
\pause
So that the usual estimate will be too low on average and the bias is larger when the underlying rates are larger. 
\end{frame}

\begin{frame}
\frametitle{Variance}
$
\mathbb{V}(\bar{Y}) = \frac{\lambda(1-\phi)(1+\phi\lambda)}{n}
$
Using the conditional variance formula you can show the variance formula above.
\end{frame}

\begin{frame}
\frametitle{Mean squared error}
Using the bias variance decomposition of mean squared error:

$
\text{MSE} = \phi^2\lambda^2 + \frac{\lambda(1-\phi)(1+\phi\lambda)}{n}
$
So that even if you take a large number of measurements ($n\to \infty$) your conclusion will (on average) be incorrect. 
\end{frame}

\begin{frame}
\frametitle{How to decide which model?}
When to use the zero-inflated Poisson? Use the score test:\\ \vspace{0.25in}
\pause
$
S = \frac{(n_0 - np_0)^2}{np_0(1-p_0) - n\bar{y}p_0^2},
$
\vspace{0.25in}
\pause
Here $n_0$ is the observed number of zeroes and $p_0$ is the estimated probability of a observed zero in a Poisson distribution. This test statistic is distributed $\chi^2_1$ if the null hypothesis is true (Poisson). 
\end{frame}

\begin{frame}
\frametitle{Example}
Summary Statistics table of $n_0$, $\hat{\lambda}$, and $\bar{y}$.  \\

\begin{table}

\begin{center}
\begin{tabular}{c | c l c  c  c  c  }
\hline
\multicolumn{6}{c}{Species} \\
\cline{1-6}
& stunt & spiral & ring  & dagger & rootknot \\
\hline
$n$& 144 &144  &144 &144 &144  \\
$n_0$ &54  & 12 &135 &59 & 121 \\
$\bar{y}$ &85.93 &264.72  &2.92 & 32.08& 12.33 \\
$S$ &$3.83\times10^{38}$ &$9.28\times10^{114}$ & $2.64\times10^3$ &$2.07\times10^7$ &$6.28\times10^7$  \\
\hline
\end{tabular}
\end{center}
\caption{Summary statistics for example zero-inflated Poisson analysis. }
 \label{tab:table_zip_data}
\end{table}

\end{frame}

\begin{frame}
\frametitle{Example (cont)}
Estimates under each regime and observed biases
\begin{table}

\begin{center}
\begin{tabular}{c | c l c  c  c  c  }
\hline
\multicolumn{6}{c}{Species} \\
\cline{1-6}
& stunt & spiral & ring  & dagger & rootknot \\
\hline
$n$& 144 &144  &144 &144 &144  \\
$n_0$ &54  & 12 &135 &59 & 121 \\
$\bar{y}$ &85.93 &264.72  &2.92 & 32.08& 6.38 \\
$\hat{\lambda}_z$& $86.79$ & 264.72$$ & $5.99$ & $32.46$ & $7.49$  \\
$\hat{\phi}$ & $1.1\%$ & $0\%$ & $51.3\%$ & $1.1\%$ & $14.7\%$  \\
bias & $-0.95$ & $0.00$ & $-3.07$ & $-0.38$ & $-1.10$  \\
\hline
\end{tabular}
\end{center}
\caption{Summary statistics for example zero-inflated Poisson analysis. }
 \label{tab:table_zip_data}
\end{table}

\end{frame}

\begin{frame}
\frametitle{Data}
\begin{figure}[htb]
\includegraphics[scale=0.35]{cc_stunt.jpg}
\end{figure}
\end{frame}

\begin{frame}
\frametitle{Data}
\begin{figure}[htb]
\includegraphics[scale=0.35]{dagger.jpg}
\end{figure}
\end{frame}

\begin{frame}
\frametitle{Data}
\begin{figure}[htb]
\includegraphics[scale=0.35]{stunt_zero_prob.jpg}
\end{figure}
\end{frame}

\begin{frame}
\frametitle{Data}
\begin{figure}[htb]
\includegraphics[scale=0.35]{dagger_zero_prob.jpg}
\end{figure}
\end{frame}

\begin{frame}
\frametitle{Conclusions I}
\begin{itemize}
\item Test for zero-inflation
\item Use the appropriate model
\item Incorrect model leads to understating impact
\end{itemize}
\end{frame}

\begin{frame}
\frametitle{Conclusions II}
Policy implications
\begin{itemize}
\item Incorrect model leads to understating impact 
\item You will never collect enough data to correct the error
\item The bias is proportional to the rate so larger rates imply larger bias 
\end{itemize}
\end{frame}

\begin{frame}
\frametitle{Conclusions III}
Policy implications
\begin{itemize}
\item Incorporating covariates into the zero-inflation can improve understanding and prediction
\item Similar form of score test to determine which model is appropriate
\item Many variations to estimate the ZIP regression model
\end{itemize}
\end{frame}

\section{Regression models}

\begin{frame}
\frametitle{How to decide which model?}
If you have covariates in a Poisson regression model the test becomes
$
\frac{\sum_{i=1}^n(\mathbb{I}(y_i=0)\exp(\lambda_i) - 1)^2}{(\sum_{i=1}^n\mathbb{I}(y_i=0)\exp(\lambda_i) - 1 ) - \vec{\hat{\lambda}}^T \hat{H}\vec{\hat{\lambda}}},
$
Here $\hat{H}$ is the projection matrix in the Poisson regression, the $\lambda_i=\exp(\vec{\hat{\beta}\vec{x}_i})$ and $\mathbb{I}(y_i=0)$ is an indicator function taking the value 1 if the condition in the parenthesis is true. 

\end{frame}


\begin{frame}
\frametitle{ZIP Regression models I}
Many approaches to estimation:
\begin{itemize}
\item Maximum Likelihood
\item EM algorithm
\item Monte Carlo EM
\item Markov chain Monte Carlo (Metropolis-Hastings)
%\item Gibbs sampling?
%\item Variational Bayes?
\end{itemize}
\end{frame}

\begin{frame}
\frametitle{ZIP Regression models II}
Typically all approaches assume: 
$\text{log}(\phi/(1-\phi)) = \vec{g}^T\vec{\gamma}$ 
and \vspace{0.35in}
$\text{log}(\lambda)=\vec{x}^T\vec{\beta}$
of course other links are possible but these are the most common. 
\end{frame}

\begin{frame}
\frametitle{Maximum Likelihood}
Substitute equations from previous slide into 
$$ \Pr(Y=y)=
\begin{cases}
(1-\phi)\exp(-\lambda)\lambda^Y/Y!, &\text{if }y\neq0 \\
\phi+(1-\phi)\exp(-\lambda), &\text{if }y=0
\end{cases}
$$
Here $1 > \phi > 0$ and $\lambda >0$.
Challenge is non-linear optimization is difficult, may not converge. No standard software implementations exist. 
\end{frame}

\begin{frame}
\frametitle{EM algorithm}
Iterate between expectation step and optimizing step. Optimizing step is two weighted regressions: 
 \begin{enumerate}
 \item Poisson with weights $1-w_i$
 \item Logistic with weights $(1-w_i, w_i)$, $Y^* = (y_i, y_{i_1}, \dots, y_{i_{n_0}}$ similar for covariates $g_i$ 
\item $W_i = (1+\exp(-\text{log}(\phi_i/(1-\phi_i)) -\lambda_i))^{-1}$ for $i=1, \dots, n$
\end{enumerate}
\end{frame}

\begin{frame}
\frametitle{MC-EM algorithm}
Iterate between MC step and optimizing step. Optimizing step is two regressions: 
 \begin{enumerate}
 \item Poisson regression with zeroes where $Z=0$
 \item Logistic regression with $Z$ response,  
\item $Z_i \sim \text{Bern}(1+\exp(-\text{log}(\phi^t_i/(1-\phi^t_i)) -\lambda^t_i))^{-1}$ for $i=1, \dots, n$
\end{enumerate}
\end{frame}

\begin{frame}
\frametitle{MH algorithm}
Iterate the following steps. 
 \begin{enumerate}
 \item propose $\vec{\beta}^t\sim \text{Normal}(\vec{\beta}^{t-1},\sigma_{\beta}^2)$
 \item accept $\vec{\beta}^t$ w/prob $1 \wedge \left(\prod_{i=1}^n\frac{\text{Pois}(Y_i \vert e^{\vec{x}_i\vec{\beta}^t}) \pi(\vec{\beta}^t) }{\text{Pois}(Y_i \vert e^{\vec{x}_i\vec{\beta}^{t-1}})\pi(\vec{\beta}^{t-1})} \right)$
  \item propose $\vec{\gamma}^t\sim \text{Normal}(\vec{\gamma}^{t-1},\sigma_{\gamma}^2)$
\item accept $\vec{\gamma}^t$ w/prob $1 \wedge \left(\prod_{i=1}^n\frac{\text{Bern}(Z_i^{t-1} \vert p_i) \pi(\vec{\gamma}^t) }{\text{Bern}(Z_i^{t-1} \vert p_i)\pi(\vec{\gamma}^{t-1})} 
\right)$, where $p_i=(1+e^{\vec{\gamma}^{t-1}g_i})^{-1}$
\item $Z_i^t \sim \text{Bern}(1+\exp(-\vec{\gamma}^{t}\vec{g}_i -\lambda^t_i))^{-1}$ for $i=1,\dots,n$
\end{enumerate}
\end{frame}


\end{document} 

